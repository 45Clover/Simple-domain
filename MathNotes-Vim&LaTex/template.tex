
%!TEX program = xelatex

\documentclass[11pt]{article}
\usepackage[a4paper, margin=1in]{geometry}
\usepackage{amsmath, amssymb}
\usepackage{tcolorbox}
\usepackage{lmodern}
\usepackage{fancyhdr}
\usepackage{fontspec} % Needed for emoji + Unicode
\setmainfont{Latin Modern Roman} % or something like DejaVu Serif

\setlength{\headheight}{14pt}
\pagestyle{fancy}
\fancyhf{}
\fancyhead[L]{Math Notes}
\fancyhead[R]{\today}
\fancyfoot[C]{\thepage}

\tcbuselibrary{skins, breakable}

% Custom boxes with emojis
\newtcolorbox{examplebox}[1][]{colback=blue!5!white, colframe=blue!80!black, title=📦 \textbf{Example}, breakable, sharp corners, #1}
\newtcolorbox{keynotebox}[1][]{colback=yellow!10!white, colframe=orange!80!black, title=📌 \textbf{Key Idea}, breakable, sharp corners, #1}
\newtcolorbox{funbox}[1][]{colback=green!5!white, colframe=green!60!black, title=🤡 \textbf{Silly Trick}, breakable, sharp corners, #1}

\begin{document}

\section*{Limits}

Here's an intro to limits.

\begin{examplebox}
Find \( \lim_{x \to 2} x^2 \).  
Solution: \( \lim_{x \to 2} x^2 = 4 \)
\end{examplebox}

\begin{keynotebox}
A limit describes the behavior of a function as it approaches a point.
\end{keynotebox}

\begin{funbox}
“Get closer without kissing!” – Think of the limit as approaching, not touching.
\end{funbox}

\end{document}

\documentclass[11pt]{article}
\usepackage[a4paper, margin=1in]{geometry}
\usepackage{amsmath, amssymb}
\usepackage{tcolorbox}
\usepackage{lmodern}
\usepackage{fancyhdr}
\usepackage{fontspec} % Needed for emoji + Unicode
\setmainfont{Latin Modern Roman} % or something like DejaVu Serif

\setlength{\headheight}{14pt}
\pagestyle{fancy}
\fancyhf{}
\fancyhead[L]{Math Notes}
\fancyhead[R]{\today}
\fancyfoot[C]{\thepage}

\tcbuselibrary{skins, breakable}

% Custom boxes with emojis
\newtcolorbox{examplebox}[1][]{colback=blue!5!white, colframe=blue!80!black, title=📦 \textbf{Example}, breakable, sharp corners, #1}
\newtcolorbox{keynotebox}[1][]{colback=yellow!10!white, colframe=orange!80!black, title=📌 \textbf{Key Idea}, breakable, sharp corners, #1}
\newtcolorbox{funbox}[1][]{colback=green!5!white, colframe=green!60!black, title=🤡 \textbf{Silly Trick}, breakable, sharp corners, #1}

\begin{document}

\section*{Limits}

Here's an intro to limits.

\begin{examplebox}
Find \( \lim_{x \to 2} x^2 \).  
Solution: \( \lim_{x \to 2} x^2 = 4 \)
\end{examplebox}

\begin{keynotebox}
A limit describes the behavior of a function as it approaches a point.
\end{keynotebox}

\begin{funbox}
“Get closer without kissing!” – Think of the limit as approaching, not touching.
\end{funbox}

\end{document}

