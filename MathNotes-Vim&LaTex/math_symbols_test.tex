% !TEX program = xelatex 
\documentclass[11pt]{article}
\usepackage[a4paper, margin=1in]{geometry}
\usepackage{amsmath, amssymb}
\usepackage{tcolorbox}
\usepackage{fancyhdr}
\usepackage{fontspec} % Required for emojis and Unicode
\setmainfont{Latin Modern Roman}

\pagestyle{fancy}
\fancyhf{}
\fancyhead[L]{Math Symbols Test}
\fancyhead[R]{\today}
\fancyfoot[C]{\thepage}
\setlength{\headheight}{14pt}

\tcbuselibrary{skins, breakable}
\newtcolorbox{examplebox}[1][]{colback=blue!5!white, colframe=blue!80!black, title=📦 \textbf{Example}, breakable, sharp corners, #1}
\newtcolorbox{keynotebox}[1][]{colback=yellow!10!white, colframe=orange!80!black, title=📌 \textbf{Key Idea}, breakable, sharp corners, #1}
\newtcolorbox{funbox}[1][]{colback=green!5!white, colframe=green!60!black, title=🤡 \textbf{Silly Trick}, breakable, sharp corners, #1}

\begin{document}

\section*{Symbol Testing}

\begin{examplebox}
Evaluate the definite integral:
\[
\int_0^1 x^2 \, dx = \left[ \frac{x^3}{3} \right]_0^1 = \frac{1}{3}
\]
\end{examplebox}

\begin{keynotebox}
In general, for any positive integer \( n \),
\[
\int_0^1 x^n \, dx = \frac{1}{n+1}
\]
\end{keynotebox}


\begin{examplebox}
Test of summation symbols:
\[
\sum_{n=1}^{\infty} \frac{1}{n^2} = \frac{\pi^2}{6}
\quad \text{ and } \quad
\sum_{k=0}^{10} k = \frac{10 \cdot 11}{2} = 55
\]
\end{examplebox}

\begin{funbox}
To remember the Gaussian sum trick:
\[
\sum_{k=1}^{n} k = \frac{n(n+1)}{2}
\]
Gauss thought of summing pairs: \(1 + n\), \(2 + (n-1)\), etc.
\end{funbox}

\begin{examplebox}
Limit identities:
\[
\lim_{x \to 0} \frac{\sin x}{x} = 1 \quad \text{and} \quad \lim_{x \to \infty} \frac{1}{x} = 0
\]
\end{examplebox}

\begin{keynotebox}
You can use epsilon-delta definitions to prove limits rigorously.
\end{keynotebox}

\newpage

\section*{More Tests}

\begin{examplebox}
Testing square roots and inline math:

\textbf{Inline:} The square root of 2 is \( \sqrt{2} \approx 1.41 \)

\textbf{Display:}
\[
\sqrt{x^2 + 1} = \sqrt{25} = 5
\]
\end{examplebox}

\begin{examplebox}
Math sets and Greek letters:
\[
x \in \mathbb{R}, \quad \alpha + \beta = \theta, \quad \phi^2 = 1
\]
\end{examplebox}

\begin{examplebox}
Aligned equations:
\[
\begin{aligned}
f(x) &= x^2 + 2x + 1 \\
     &= (x+1)^2
\end{aligned}
\]
\end{examplebox}

\begin{keynotebox}
Use \texttt{aligned} inside \texttt{equation} or \texttt{display math} for multi-line math.
\end{keynotebox}

\begin{funbox}
Remember: $\pi r^2$ is not pie are square, it's **area**!  
Pie are round 🤡
\end{funbox}

\end{document}

